%TODO
% spell check, b.v. ispell op de uva
% zoeken op TODO punten

%% vi: set tabs top=2, set textwidth=80
\documentclass[a4paper,11pt]{article}

%\usepackage{homework}
% \usepackage{graphicx, subfigure}
% \usepackage{verbatim}
% \usepackage{algorithm, algorithmic}
% \usepackage{amsmath, amsthm, amssymb}
\usepackage[english]{babel}
\usepackage{url}

\title{Mean shift object tracking\\ Lab report for IMS}
\date{}
\begin{document}
\maketitle 

\section{Introduction}
	Videos are ubiquitous; in the United Kingdom it has been claimed that there is one observation camera per 14 citizens \footnote{
Rt Hon David Davis MP, June 12 2008, speech:"It is incumbent upon me to take a stand" 
\url{http://www.conservatives.com/News/Speeches/2008/06/David_Davis_It_is_incumbent_upon_me_to_take_a_stand.aspx}
}. YouTube.com, the largest video website in the USA with a marketshare of 40\% of all watched online videos serves more than 10 Billion movies a month\footnote{comScore Video Metrix, August 2009 
\url{http://comscore.com/Press_Events/Press_Releases/2009/9/Google_Sites_Surpasses_10_Billion_Video_Views_in_August}}.
	Furthermore, the amount of videos uploaded to the site through mobile phones is growing exponentially \footnote{The Official YouTube Blog, June 25 2009, blogpost:"Mobile Uploads to YouTube Increase Exponentially" 
\url{http://youtube-global.blogspot.com/2009/06/mobile-uploads-to-youtube-increase_5122.html}}. 
	From such facts, it is clear that the increasing signifance of video content, ranging from the practical to recreational, is undeniable. This plethora of video content makes automated video analysis desirable. Ideally we would like to analyse videos to be able to extract their content. This includes making them accessible for search or to alert the viewer of important developments. One of these analyses is tracking an object through multiple frames of video.

Generally, tracking can be done in two distinct manners; model based and appearance based. 
The former approach begins by building a substantial descriptive model of the object to be tracked. One can think of a wireframe model of the object together with texture or color information. In every frame this model is searched for, and the location that resembles the model best is the location of the relevant object.  A problem with this approach is that a substantial descriptive model has to be constructed for every conceivable object to be tracked.  Because of this tedious task, this approach can only be applied to small specialized domains.  

This paper, however, will analyse an instance of the latter, appearance based, approach. This approach does not build a full model for tracking, but rather it looks at the appearance of an object and searches for regions that appear the same.  This is thus not a substancial descriptive model; rather it is an appearance based model. The underlying assumption behind this way of tracking is that small local features of the tracked object appear the same over multiple frames. %One such feature is color.
The appearance based tracker discussed in this paper is the mean-shift tracker. 

Color is the feature that will be tracked, so different color models will be presented and applied. Following this, subsequent tracking results on different videos for both trackers will be discussed. Specifically, we want to explain through experiments how the mean-shift tracker behaves under varying video conditions (such as occlusion and objects of varying scales), and what the influence of the used color model is.
	As said at the beginning of this section, videos are pervasive, and their analysis should therefore ideally be done in realtime. The speed of the tracker will, therefore, also be discussed. 
	
	This paper is organised as follows:  after the introduction of the topic in section 1, the theory behind the used trackers and color models will be discussed in section 2.
	Section 3 will be devoted to the implementation of the presented theories. The experimental setup, a description of the data, subsequent results and their discussion will be presented in section 4. 
	Finally, section 5 will present the main conclusions on color based mean shift tracking.

\section{Theory}	
	This section will explain the main theories used in the mean shift tracker. The color of an object is dependent on what we can call the intrinsic color of the object and other factors such as lighting. Because the tracker will track an object on its color, it is important to choose a color model that is invarient to all factors except the instrinsic color of the object. Therefore in the following section, color models in general as well as some specific color models and their invarient properties will be introduced. Following this, histograms, which are probabilistic descriptions, or probability density functions (PDF), of color in image regions, will be discussed. We can compare PDFs using certain distance measures, which enables us to find color similarities between image regions. These concepts are integral to understanding the mean-shift tracker, which will be discussed afterwards. For an interesting comparison, the simpler brute force tracker willl be briefly discussed last.  

%DONE TILL HERE

\subsection{Color models}
%		It was previewed in the introduction that an object would be tracked by its appearance, and that the feature used to do so would be color. In this section we will describe three different color models and their relevance to appearance based tracking, but first we will discuss some general properties of color.

Before we start this section it is important to make note of some color characteristics. First there is a difference between what we have called the intrinsic color of an object and the appearent color of that object. As an example, a white car will look white in normal daylight, however, when lit at night by the orange light of a sodium lamp the white car will look orange. In this example the intrinsic color of the car is white, while the appearent color is either white or orange depending on the lightsource. In the following discussion on light we will explain this color behavior. Furthermore, an object rarely has uniform color, therefore an object patch is defined as a sufficiently small region of an object such that it has uniform color properties.

When a camera is aimed at an object patch, the color it detects is dependent on the Spectral Power Distribution (SPD) of the light source, the reflectance factor of a patch of the object, the color matching functions of the camera sensors and the geometric arrangement of the lightsource(s), the object and the observer, as well as the geometric properties of the object itself. The SPD describes for the whole visible light spectrum with what intensity every wavelength is being radiated. The reflectance factor describes how much each part of the spectrum gets reflected and how much gets absorbed by the object patch, this is the source of what we have called . The color matching functions indicate the magnitude of the response of a camera sensor for every wavelength of incoming radiation. Last, the geometric arrangement and properties of the object also influences the color \cite{gevers_color}.

It is important to distinguish between the reflectance factor of a patch of an object, and the appearance of this patch's color to a camera. All factors described above influence the appearance of the patch, however the reflectance factor stays constant. Therefore, we want to track an object based on its reflectance factor since this is the only factor intrinsic to the object.  Hence, we want to have a color model which is ideally invariant to all factors that influence the appearance of the object except for the reflectance factor.  Last color model and color model will be used interchangably, 

\subsubsection{RGB and XYZ}
		
All percievable colors can, according to trichromacy theory \cite{gevers_color}, be modeled as the response of three sensors, namely: red, green, blue. The RGB color model resembles the human perception of colors in the sense that human color vision can also be said to be the result of a tri-stimulus.  RGB is an additive color model, which describes each color as a mixture of the three mentioned primary colors.

Some properties of the RGB color model is that it is a device dependent color model, meaning that a given RGB color will be detected or reproduced differently on different devices. This behavior is caused by the fact that the three primary colors are not defined and therefore every device might use three different pimary colors to detect or reproduce colors. Another property of RGB is that it is not invariant to illumination intensity, illumination intensity, highlights nor geometry \cite{gevers_invariant}. Since RGB is not invariant to any of the factors which influence the appearant color of an object patch, this color model seems ill equiped for tracking. However it is the standard model for representing colors and therefore woth mentioning.

the XYZ color model, just like RGB is also a tri-stimulus model, however it is defined  with 3 fictive primary colors X, Y and Z. The three colors are a linear combination of the RGB colors. However, the X,Y and Z colors are well defined, therefore, the XYZ colorspace is device independent. All invariant properties are the same for both RGB and XYZ.

%which means that if the amount of light that falls on an object changes, th appearance changes, the human analog is the perception of shades. 
%RGB is a standard colorspace used in computer vision, which desribes the detected colors as a tri-stimulus, (R,G,B). Where R corresponds the response of the Red sensor, which is most sensitive to Red, G to the response of the green sensor and B to the resonse of the Blue sensor.
%However, this model has some problems for the domain of object tracking. it is not invariant to illumination intensity, which means that if the amount of light that falls on an object changes, th appearance changes, the human analog is the perception of shades. 
		

		
\subsubsection{Normalised RGB and normalised XYZ}

In the RGB color model a change in lighting intensity would make the responses of all three color channels increase proportional to the intensity change. Through normalisation the lighting intensity can be fixed, which makes normalised RGB(rgb) invariant to illumination intensity. The geometric arrangements also influences all color channels proportionally and therefore influnece the color intensity of the perceived object patch. Invariance to illumination intensity therefore also creates invariance for geometric arrangement. rgb however is still not invariant to illumination color or highlights \cite{gevers_invariant}. 
Let $I$ denote the intensity, then rgb can be calculated from RGB in the following way:
\begin{eqnarray}
\label{eq:rgb}
I &=& R+G+B \\
r &=& \frac{R}{I},\; g = \frac{G}{I},\; b = \frac{B}{I}
\end{eqnarray}

Two things are worth concidering here, first $r,g,b$ are undefined for black when $I=0$, this can be remedied by recognising that white black and grey are the same except for intensity. Thus black in rgb space should be the same as white or grey which is $r=g=b$. Last the rgb colorspace is 3-dimensional, however since all 3 channels are divided by the intensity, it must be that $r+g+b=1$. Therefore, rgb it is overdetermined as a 3-dimensional space and can be described in 2 dimensions $r,g$, where $b$ can be deduced through $b=1-(r+g)$.

Similar to RGB the device independent colorspace XYZ can be normalised to a normalised XYZ(xyz).		
		
\subsubsection{HSV}
The hue, saturation, value model(HSV) does not represent color as a tri-stimulus, rather it uses three color properties to describe a color. hue describes the dominant wavelength, saturation describes how dominant this wavelength is with respect to the whole of the visual spectrum, last value indicates the brightness of the color response, similar to Intensity. The HSV-space can be thought of as a cone, where the central axis ranges between black at the bottom and white on top, the value indicates the distance along this central axis. The angle around the axis is described by the hue, while Saturation indicates the distance from the central axis to the edge of the cone. 

When we defenestrate all but the Hue, we are left with a 1-dimensional space, describing color in a manner that is invariant to illumination and geometrical conciderations as well as highlights \cite{gevers_invariant}. The illumination intensity invariance can be understood in the following way, by removing the value we make our representation illumination independent, since value exactly describes this property. We can derive hue $H$ from RGB as follows:

\begin{equation}
\label{eq:hue}
H = arctan\left(\frac{\sqrt{3}\,(G-B)}{(R-G)+(R-B)}\right)
\end{equation} 
		 
\subsection{Histogram}
		We want to describe an image patch as a probability density function(pdf). This means that we could compare different image patches and a measure of similarity would be how much both distributions look alike %TODO uglysentence thatis
		Furthermore we could compare patches of different size since the probability distribution would describe to what relative frequency colors are in a patch.
		Ideally the PDF  would take every color and describe its relative frequency.However this would lead to a pdf which is too sparse. Therefor we collect similar colors in bins and compare the relative frequencies of this more coarse color description. A problem with this is that too few bins would lead to a pdf which is too coarse. Therefore finetuning the number of bins is a delicate task, of balancing between representations which are too coarse and too sparse, to be useful. 
\subsubsection{Epanechnikov Weighting}
\subsubsection{Bhattacharyya Distance}
		
\subsection{Meanshift}
Mean shift is an algorithm that finds the most probable target position of a tracked object in a frame. First a the tracked object is modeled by computing a color histogram. Likewise on the same location but in the next frame an other color histogram is computed. These two histograms are compared using a Bhattagarya distance measure. This results in a weight image where high values represent pixels that are similar.  For the intuition, when an object moved to the left, the weight values will be high on the left side. Furthermore the weights will be low on the right side because new information (different colors) entered the search window. 
This is used to weight an image which contain relative location (where the origin is situated in the center of this image). This result in an image where each pixel represents a weighted shift. It is easy to see that the mean shift is the mean of this image.

\subsection{Brute force}
So we are talkign about this because it is interstingly silly. We want to say meanshift tracker is less computationally complex and therefore faster. 
First the tracked object is modeled by computing a color histogram; the target model. In the next frame a new color histogram is computed on every location; these are the candidate models. The target model is compared with all candidate models using a distance measure. The new location is determined by the location of the candidate histogram with the lowest distance.
		
\subsubsection{Semi Brute force}
Using the fact that we work with 25 frames every second and objects are moving within a maximum speed, an assumption can be made that the object location should be close to the previous location. This optimization ignores a lot of unnecessary computations.

	

	
	% TODO 
	%Above steps are done iteratively till a 

	% (2) is to be sure pixels that are farther away from the center are assigned a lower weight.

	% To prevent overshoot the mean shift vector is divided by 2 and added up to
	% the old location Y1.

	% The algorithm repeats on this new location until the mean shift vector is
	% smaller then a certain convergence threshold, for example 1 pixel.

\section{Implementation}

\section{Results} 
	\subsection{Experimental setup} 
	binsize is per colorchannel
	\subsection{Brute force} 
	\subsection{Meanshift} 

	\begin{tabular}{l*{8}{r|}}
		scene	& 	colorspace	& downsampled & keeped track with binsize & 4 & 8 & 16 & 32\\
		\hline
		soccer 	& 	RGB	 		& 1x		  &			bla				  & + & + & +  &  -\\
		soccer 	& 	RGB	 		& 1x		  &			bla				  & + & + & +  &  -\\
		soccer 	& 	rg	 		& 1x 		  &			bla				  & + & + & +  &  -\\
		soccer 	& 	H	 		& 1x		  &			bla				  & + & + & +  &  -\\
		soccer 	& 	HS	 		& 1x		  &			bla				  & + & + & +  &  -\\
		soccer 	& 	HSV	 		& 1x		  &			bla				  & + & + & +  &  -\\
		soccer 	& 	RGB	 		& 2x		  &			bla				  & - & - & -  &  -\\
		soccer 	& 	rg	 		& 2x 		  &			bla				  & - & + & -  &  -\\
		soccer 	& 	H	 		& 2x		  &			bla				  & - & + & -  &  -\\
		soccer 	& 	HS	 		& 2x		  &			bla				  & - & + & -  &  -\\
		soccer 	& 	HSV	 		& 2x		  &			bla				  & - & - & -  &  -\\
	\end{tabular}	

\subsubsection{Conclusions}

	%TODO compare with other scene
	%make the comparison automatic, use wrapper function and write figure
	%windows to file with binsize etc. as filename
\subsection{Discussion} % discussion of results
performs bad on low binsize (obvious) and downsampled images (obvious)\\

performs bad on a very high (32) a binsize even with not downsampled
images, the high bin size causes to lose generalisation power (same object other pose)\\
		
colorspace RGB and HSV only keeped track with a non downsampled image,
that is probably because it uses intensity i


\section{Conclusions} \label{sec:conc}
\section{Future Work} \label{sec:fut}
Observe how you can improve your design and them describe how you implemented this change or, for lack of time, describe how you would change your design. 

e.g. a different color space or the number of bins in the histogram
test the tracker also on a video of a domain other than soccer (or any other sport on a green field)

% variable tracking window size 

%TODO write something about kalman

%TODO would be better to have a ground truth and compair the tracking
%coordinates to the ground truth, no just a binary val (keeped track or not) is
%used


\section{References} 

% \begin{figure}[!ht]
% \centering
% \includegraphics[height=7cm]{img/fprate}
% \caption{The false positive rate per layer, averaged over the test set.}
% \label{fig:fprate}
% \end{figure}

\renewcommand\bibname{References}
\bibliography{references}
\bibliographystyle{IEEEtran}
\end{document}
