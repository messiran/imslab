%TODO
% spell check, b.v. ispell op de uva
% zoeken op TODO punten

%% vi: set tabs top=2, set textwidth=80
\documentclass[a4paper,11pt]{article}

\usepackage{homework}
% \usepackage{graphicx, subfigure}
% \usepackage{verbatim}
% \usepackage{algorithm, algorithmic}
% \usepackage{amsmath, amsthm, amssymb}
\usepackage[english]{babel}

\title{Mean shift object tracking\\ Lab report for IMS\\ Tjerk Kostelijk, Bart Buter\\{mailtjerk, bjbuter}@gmail.com}

\date{December 22, 2009}

\begin{document}
\maketitle
\section{Introduction}
\section{Theory}
	\subsection{Colorspace}
	\subsection{Histogram}
	\subsection{Brute force}
	\subsection{Meanshift}

\section{Implementation}

\section{Results} 
	\subsection{Experimental setup} 
	different colorspaces
	compare meanshift with brute force
	images etc.
	\subsection{Brute force} 
	\subsection{Meanshift} 

\subsection{Discussion} % discussion of results


\section{Conclusions} \label{sec:conc}
\section{Future Work} \label{sec:fut}
Observe how you can improve your design and them describe how you implemented this change or, for lack of time, describe how you would change your design. 

e.g. a different color space or the number of bins in the histogram
test the tracker also on a video of a domain other than soccer (or any other sport on a green field)

\section{References} 

% \begin{figure}[!ht]
% \centering
% \includegraphics[height=7cm]{img/fprate}
% \caption{The false positive rate per layer, averaged over the test set.}
% \label{fig:fprate}
% \end{figure}

% \renewcommand\bibname{References}
% \bibliography{references}
% \bibliographystyle{IEEEtran}
\end{document}
